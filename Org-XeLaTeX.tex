% !TEX program = xelatex
\documentclass{ctexrep}

\usepackage{fontawesome}
\usepackage{ecnu}
\author{张yadong}
\date{\today}
\title{人文社科类学位论文写作—— \LaTeX 轻量级应用}
\begin{document}
\maketitle
\tableofcontents



\section{写作缘由}
\label{sec:orgddf61dc}
\LaTeX{}的论文排版无可替代,而学校不会专门教排版设计。
针对人文社科类学位论文写作,从基本的文字编辑,数据分析,进度管理,版本控制到最后的排版设计,笔者写下这篇文章,
希望能够帮助大家完成自己的
毕业论文。
笔者默认你使用\faWindows~ Windows10操作系统,\faApple~ Mac 原理一样,Linux\ldots{}\faLinux 嗯,你应该比我知道的多。
\section{文字编辑}
\label{sec:org63c45b2}
\subsection{Org-mode 安装}
\label{sec:orgaaef9fc}
注:下列键盘符号 C 代表 Ctrl 键, M 代表 Alt 键,RET 代表回车,小写字母就是实际字母。

\begin{itemize}
\item 下载\href{https://www.gnu.org/software/emacs/download.html}{Emacs} 26.2,安装最新版 \href{https://orgmode.org/elpa.html}{Org-mode} 9.2.3,官网说的配置文件 \textbf{init file} 需要新建:
\item 打开 \textbf{runemaca.exe} ,C-x C-f 输入 \textbf{.emacs} ,新建完成
\item 按照官网介绍操作,安装完成后,M-x org-version,应该就是与官网一致的版本。
\end{itemize}
\subsection{基本操作}
\label{sec:orgad93b47}
参照国外大佬的\href{https://pan.baidu.com/s/1p6CRrnt6c0WrROvLW0BjRA }{视频(提取码:26qz )},主要记 \textbf{快捷键} (输入字母可以识别,但不是最新的,比如大写字母都改小写了),介绍顺序在\href{https://orgmode.org/worg/org-tutorials/org-screencasts/org-mode-google-tech-talk.html}{这儿}。
\subsection{特殊写作环境}
\label{sec:orgf18cb22}
写作前,开头加入这段文本:
\begin{verbatim}
# -*- coding: utf-8 -*-
\end{verbatim}

保证你的文字导出时不会乱码。

1.摘要

\begin{verbatim}
#+begin_abstract
摘要测试
#+end_abstract
\end{verbatim}

2.代码

C-c C-,

默认为verbatim环境,
如果要用 minted 宏包,确认你有安装了Python包pygments,建议安装 \textbf{Anaconda} ,然后在 \#+begin
前一行加上

\begin{verbatim}
#+ATTR_LATEX: :options org-latex-minted-options
\end{verbatim}

\section{数据分析}
\label{sec:org40f6abb}
SPSS良心14天试用期够用了,如果你经常要写论文,推荐学习一下 R 语言。
\section{{\bfseries\sffamily TODO} 使用 Org-mode 的GTD工作流程}
\label{sec:org65c1eca}
\section{版本控制(可选)}
\label{sec:org567cdd4}
\subsection{安装 \href{https://git-scm.com/downloads}{Git} 2.21}
\label{sec:org8cd1a95}
\subsection{\href{https://www.liaoxuefeng.com/wiki/896043488029600}{廖雪峰的Git教程}}
\label{sec:org250cbff}
\section{协同写作}
\label{sec:org6b80c8e}
C-c C-e 导出utf-8文本,供导师修改(反正排版是最后做的)。
\section{排版设计}
\label{sec:orgf6e0dca}
笔者在设计过程中发现Org-mode中用 Xe\LaTeX 写中文文档很早就有人写配置(添加在 \textbf{.emacs} 文件的)了,
现状笔者添加到 ox-latex.el里,然后编译:M-x byte-compile-file,重新打开 runemacs.exe (推荐
添加到桌面快捷方式)就可以使用了。
所有文本写完之后,跳到开头,C-c C-e \#,输入latex,应该可以看到一下信息:

\begin{verbatim}
#+latex_class: ctexrep
#+latex_class_options:<默认为[12pt, a4paper],可以自己设置>
#+latex_header:{\usepackage{学校 \LaTeX 模板样式(.sty结尾),我用的 ecnu}
#+latex_header_extra:
#+description:
#+keywords:
#+subtitle:
#+author: 
#+title: <默认显示你的文件名>
#+latex_compiler: xelatex
#+date: \today 
\end{verbatim}

author,title,date可以删掉,如果有学校封面。

表格,图片一律使用图片导入:C-c C-l file,选择你的图片路径即可。
示例:
\begin{center}
\includegraphics[width=.9\linewidth]{org-mode-unicorn-logo.png}
\end{center}
\subsection{安装 \href{https://zhuanlan.zhihu.com/p/64555335}{\TeX{} Live}(强烈推荐)2019}
\label{sec:orgfd45bc8}
安装完成后,打开你的 \textbf{.org} 文件,C-c C-e l o,即可打开编译好的 \textbf{pdf} 文件,当然可以自己改一下
\textbf{.tex} 文件,这样排版工作量就会少很多。
\section{参考链接}
\label{sec:orgb91b710}
\subsection{{\bfseries\sffamily TODO} 基于 gbt-7714-2015 格式参考文献编译}
\label{sec:org4da2f8a}

[1] \url{https://www.reddit.com/r/emacs/comments/4k1lp2/noob\_question\_how\_to\_set\_locales\_and\_encoding\_for/}

[2] \url{https://www.cnblogs.com/wangkangluo1/archive/2012/02/04/2337705.html}

[3] \url{http://www.cnblogs.com/visayafan/archive/2012/06/16/2552023.html}

[4] \url{https://xiaoguo.net/wiki/org-mode-book.html}

[5] \url{https://orgmode.org/manual/index.html\#SEC\_Contents}

[6] \url{https://orgmode.org/worg/org-tutorials/org-screencasts/org-mode-google-tech-talk.html\#sec-2}

[7] \url{https://zhuanlan.zhihu.com/p/64555335}
\end{document}
